\section{Fazit (Joshua Hörmann)}
Die Entwicklung des ArtBot hat uns einen guten Einblick in die Entwicklung von Chatbots gegeben. Dabei sind einige kleinere Schwierigkeiten aufgetreten. So wollten wir zuerst die Bilder in die Datenbank speichern, was allerdings dazu führte, dass die Datenbank sehr groß wurde und es zu diversen Problemen bei der Umwandlung kam. Da Bilder in einer Datenbank nur als BLOB (Binary Large Object) gespeichert werden können und daher immer wieder richtige umgewandelt werden müssen, wenn man sie sich aus der Datenbank holt. Zusätzlich ist es sehr zeitaufwendig die nötigen Testdaten zu schreiben, was uns bereits bei unserer Recherche bezüglich AIML aufgefallen ist. Glücklicherweise haben wir hier Chatito gefunden, was hierbei Abhilfe schaffte. Ebenso wäre das Erstellen einer GUI und die Kommunikation dieser mit Rasa mit erheblichem Aufwand verbunden gewesen, weshalb wir hierbei auf das Chatbot Widget zurückgegriffen haben. \\
Für die Weiterentwicklung des ArtBots kann man auf die in Kapitel \ref{sec:erweiterungen} genannten Erweiterungen verweisen. Abschließend ist zu sagen, dass der Einstieg in Rasa recht simpel war, also das Erstellen des ersten einfachen Chatbots. Allerdings wurde es in dem Moment, wo wir uns auf unseren Anwendungsfall konzentrierten, deutlich komplexer.