\section{Einführung (Joshua Hörmann)}
Chatbots sind heutzutage allgegenwärtig, bei diversen Webseiten springt einem direkt ein Chatfenster entgegen, welches sich erkundigt ob es nicht behilflich sein kann. Deshalb haben wir uns bei unserem Projekt im Fach Anwendung der KI für das Thema Conversational AI entschieden. \\
Bei der von uns entwickelten  Conversational AI handelt es sich um einen Chatbot, welcher Bilder von bekannten Künstlern ausgeben kann. Dieser Chatbot wird im weiteren Verlauf ArtBot genannt. Das Ziel des Projektes war es einen funktionierenden Chatbot zu erstellen, welcher möglichst zuverlässig auf unterschiedliche Anfragen reagiert und die passenden Bilder zu den jeweiligen Anfragen zurück gibt. Zu diesem Zweck wurde das Rasa Framework verwendet, welches mit einigen Tools und Libraries erweitert wurde. Da die Standardimplementierung von Rasa nur auf der Kommandozeile funktioniert und dort keine Bilder ausgegeben werden können, musste der ArtBot um eine GUI ergänzt werden. Ebenso wurde ein Tool verwendet, um das Erstellen der Trainings- und Testdaten zu vereinfachen. \\
Diese Dokumentation beschäftigt sich zuerst mit der Kontextabgrenzung und der Datengrundlage, daraufhin werden das Konzept und die Implementierung beleuchtet. Danach wird der Workflow des ArtBot´s und mögliche Erweiterungen diskutiert. Dann wird die Performance bewertet und zuletzt noch ein Fazit gezogen. Im Anhang befinden sich die Klassifikationsergebnisse von Rasa NLU und Rasa Core sowie die Code-Dokumentation. Diese Dokumentation beschäftigt sich vor allem mit dem Projekt im Allgemeinen und mit dem Workflow des ArtBots, weshalb einige Tools nicht bis ins Detail beschrieben werden, da dies den Rahmen sprengen würde. 


\newpage
